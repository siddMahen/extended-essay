\documentclass[a4paper,11pt]{report}

% -- Typography --
\usepackage[utf8]{inputenc}
\usepackage{microtype}
%\usepackage{parskip}

% -- Fonts --
%\usepackage{charter}
%\usepackage{palatino}

% -- Math --
\usepackage{amsmath}
\usepackage{amsthm}

% -- Tables --
\usepackage{booktabs}

% -- Graphics --
\usepackage{pgfplots}

% -- TOC --
\usepackage[nottoc,section]{tocbibind}
\usepackage[titles]{tocloft}

% -- Misc. --
\usepackage{hyperref}
\usepackage{color}
\usepackage{array}
\usepackage{url}

\begin{document}


\title{Matroids and Greedy Algorithms}
\author{Siddharth Mahendraker}

\maketitle

\begin{abstract}
This is the abstract.
\end{abstract}

% -- Bookkeeping --
\setcounter{secnumdepth}{3}
\renewcommand{\thesection}{\arabic{section}.0}
\renewcommand{\cftsecfont}{\bfseries}
\settocbibname{References}
\setlength\cftbeforesecskip{3pt}
\setlength\cftbeforesubsecskip{3pt}

% -- TOC --
\setcounter{page}{1}
\pagenumbering{roman}
\tableofcontents
\clearpage
\setcounter{page}{1}
\pagenumbering{arabic}

% -- Introduction --
\section{Introduction}

Combinatorial optimization is a mathematical framework for finding optimal
objects amongst a large set of objects. A famous example of a combinatorial
optimization problem is the traveling salesman problem, or TSP. Assume a
salesman has a list of cities he wants to vist, and knows the distance
between various pairs of cities. The TSP asks: What is the shortest path
which takes the salesman through each city once and back to his starting
point? Here, the optimal object is a shortest path which passes through each
city once and returns the salesman to his point of departure, and the set of
objects is the set of all paths which can be taken. To see why this problem
could be difficult, assume that every city is connected to every other city
and we want to check all the paths to find the shortest one. If we fix a
city of departure, there are $n-1$ cities left to go to, where $n$ is the
total number of cities. Each of these choices represents a different path
the salesman could take. In the next city, there are $n-2$ cities to left to
go to, and so on. It turns out, given $n$ cities there are $(n-1)!/2$ unique
paths which reach every city and return the salesman to his point of
departure. Even for relatively small values of $n$, the number of possible
paths is enourmous; far larger than anything even a computer could hope to
check. Consider for example the choice of $n = 64$. In this case, there are
over $10^{86}$ possible paths to check! More paths than there are atoms in
the universe! Using combinatorial optimization techniques, however, we can
significantly reduce the amount of work needed to find an optimal path.
Indeed, to date, instances of the TSP have been solved with tens of thousands
of cities.

One particular strategy often used in combinatorial optimization is the greedy
optimization strategy, or greedy algorithm. The greedy algorithm attempts to
find the optimal object by making succesive locally optimal decisions, in the
hopes that this will lead to a globally optimal solution. For example, a greedy
algorithm for the TSP could work as follows: Fix the city of departure.
To reach another city, pick the shortest route out of the current city which
leads to an unvisited city. Repeat until you return to the city of departure.
Unfortunately, greedy strategies are often non-optimal. In the case of the TSP,
this particular greedy algorithm will not yield the optimal path, and in
certain situations, can in fact give the worst possible solution. More
generally, it is known that there do not exist greedy algorithms which yield
optimal solutions to the TSP.

%% Need citations here

However, for certain combinatorial optimization problems, greedy algorithms
can be very useful. If they can be proven to be optimal, greedy algorithms
are often far simpler to implement than other combinatorial optimization
strategies, such as dynamic programming. Furthermore, they often consume
less resources when implemented on a computer and thus run faster and more
efficiently. Therefore, knowing when a greedy algorithm can be applied is
very useful.

In this paper, we examine the necessary and sufficient conditions for a
greedy algorithm strategy to be optimal. This is done through the study
of matroids and their properties, which as we shall soon see, lead naturally
to greedy algorithm strategies.

% Outline what happens in each section

\section{Matroid Theory}

\section{Greedy Algorithms and Matroids}

\section{Optimality}

\section{Example: Kruskals Algorithm}

% -- temp
\nocite{*}

\clearpage
\bibliographystyle{plain}
\bibliography{doc.bib}

\end{document}
