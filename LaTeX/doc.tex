\documentclass[a4paper,11pt]{report}

% -- Typography --
\usepackage[utf8]{inputenc}
\usepackage{microtype}
%\usepackage{parskip}

% -- Fonts --
%\usepackage{charter}
%\usepackage{palatino}

% -- Math --
\usepackage{amsmath}
\usepackage{amsthm}

% -- Tables --
\usepackage{booktabs}

% -- Graphics --
\usepackage{pgfplots}

% -- TOC --
\usepackage[nottoc,section]{tocbibind}
\usepackage[titles]{tocloft}

% -- Misc. --
\usepackage{hyperref}
\usepackage{color}
\usepackage{array}
\usepackage{url}

\begin{document}


\title{Matroids and Greedy Algorithms}
\author{Siddharth Mahendraker}

\maketitle

\begin{abstract}
This is the abstract.
\end{abstract}

% -- Bookkeeping --
\setcounter{secnumdepth}{3}
\renewcommand{\thesection}{\arabic{section}.0}
\renewcommand{\cftsecfont}{\bfseries}
\settocbibname{References}
\setlength\cftbeforesecskip{3pt}
\setlength\cftbeforesubsecskip{3pt}

% -- TOC --
\setcounter{page}{1}
\pagenumbering{roman}
\tableofcontents
\clearpage
\setcounter{page}{1}
\pagenumbering{arabic}

% -- Introduction --
\section{Introduction}

Combinatorial optimization is a mathematical framework for finding optimal
objects amongst a large set of objects. A famous example of a combinatorial
optimization problem is the traveling salesman problem, or TSP. Assume a
salesman has a list of cities he wants to vist, and knows the distance
between each city. The TSP asks: What is shortest path which brings the
salesman through each city and back to his starting point? Here, the optimal
object is an optimal path which passes through each city and returns the
salesman to his point of departure, and the set of objects is the set of
all paths which can be taken. To see why this problem could be difficult,
assume that every city is connected to every other city and we want to check 
each path to see if it is optimal. If we fix a city of
departure, there are $n-1$ cities left to choose from, where $n$ is the total
number of cities. Each of these choices indicates a different path which
could have been taken. In the next city, there are $n-2$ cities to choose
from, and so on. It turns out that given $n$ cities, there are
$\frac{(n-1)!}{2}$ unique paths which reach every city and return the
salesman to his point of departure. Even for relatively small values of $n$,
the number of choices in enourmous. Consider for example the choice of
$n = 64$. In this case, there are over $10^{86}$ possible paths to choose
from! More paths than there are atoms in the universe! Using combinatorial
optimization techniques, however, we can significantly reduce the amount of
work needed to find an optimal path. Indeed, to date, instances of the TSP
have been solved with tens of thousands of cities.

One particular strategy often used in combinatorial optimization is the greedy
optimization strategy, or greedy algorithm. The greedy algorithm attempts to
find the optimal object by making succesive locally optimal decisions. For
example, a greedy algorithm for the TSP could work as follows: Fix the city
of departure. To get to another city, pick the shortest route out of the
current city. Repeat until you return to the city of departure.
Unfortunately, greedy strategies are often non-optimal. In the case of the
TSP, this particular greedy algorithm will not yield the optimal path, and
in fact in certain situations, can give a worse path than a random algorithm
which returns a random path.

However, greedy algorithms are still useful. If they can be proven to be
optimal, greedy algorithms are often far simpler to implement than other
combinatorial optimization strategies, such as dynamic programming.
Furthermore, they often consume less resources when implemented on a computer,
and thus run faster and more efficiently. Therefore, knowing when a greedy
algorithm can be applied is very useful.

The question now becomes: Under what necessary and sufficient conditions
is a greedy algorithm optimal?

The remainder of this paper is focused on answering this question through
the use of matroid theory. In section

\section{Matroid Theory}

\section{Greedy Algorithms and Matroids}

\section{Optimality}

\section{Example: Kruskals Algorithm}

% -- temp
\nocite{*}

\clearpage
\bibliographystyle{plain}
\bibliography{doc.bib}

\end{document}
