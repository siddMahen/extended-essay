\documentclass[a4paper,11pt]{report}

% -- Typography --
\usepackage[utf8]{inputenc}
\usepackage[%
    tracking=true,%
    kerning=true,%
    selected=true,%
    stretch=20,%
    shrink=20,%
]{microtype}

% -- Misc. --
\usepackage[plain]{fancyref}
\usepackage{aliascnt}
\usepackage{hyperref}
\usepackage{color}
\usepackage{array}
\usepackage{url}

\usepackage{tikz}
\usetikzlibrary{arrows,petri,topaths}
\usepackage{tkz-berge}

\usepackage[labelfont=bf]{caption}
\usepackage{subcaption}

% -- TOC --
\usepackage[nottoc,section]{tocbibind}
\usepackage[titles]{tocloft}

% -- Math --
\usepackage[mathscr]{eucal}
\usepackage{amssymb}
\usepackage{amsmath}
\usepackage{amsthm}

\theoremstyle{plain}
\newtheorem{thm}{Theorem}[section]

% make sure counters across everything sync up
\newaliascnt{lem}{thm}
\newaliascnt{prp}{thm}
\newaliascnt{defn}{thm}
\newaliascnt{cor}{thm}

\newtheorem{lem}[lem]{Lemma}
\newtheorem{prp}[prp]{Proposition}
\newtheorem{cor}[cor]{Corollary}

% fix the definition style to bold the note text properly
\newtheoremstyle{definition}%
{}{}{\normalfont}{}{\bfseries}{.}{0.5em}%
{#1 #2\thmnote{ (#3)}}

\theoremstyle{definition}
\newtheorem{defn}[defn]{Definition}

\aliascntresetthe{lem}
\aliascntresetthe{prp}
\aliascntresetthe{cor}
\aliascntresetthe{defn}

\providecommand*{\lemautorefname}{Lemma}
\providecommand*{\thmautorefname}{Theorem}
\providecommand*{\corautorefname}{Corollary}
\providecommand*{\prpautorefname}{Proposition}
\providecommand*{\defnautorefname}{Definition}

% convinience commands
\newcommand{\I}{\mathcal{I}}
\newcommand{\B}{\mathscr{B}}

\begin{document}

\title{On the Characterization of Combinatorial Optimization Problems Soluble Using Greedy Strategies}
\author{Siddharth Mahendraker\\
    \url{siddharth.mahen@gmail.com}\\\\
    Candidate Number: 000652 -- 0016\\
    School: International School of Helsinki\\
    Word Count: 3900}

\date{November 14, 2013}

\maketitle

\begin{abstract}
In this paper, we examine the necessary and sufficient conditions required for
a greedy algorithm strategy to be successful for an arbitrary combinatorial
optimization problem. This is done by studying matroids and their properties.
We show that matroids are intimately connected to the solution of combinatorial
optimization problems using greedy strategies. Finally, we conclude that a
greedy algorithm is successful for a combinatorial optimization problem if and
only if the optimization problem has matroid structure.\\\\
Key terms: greedy algorithms, matroid theory, combinatorial optimization.
\end{abstract}

% -- Bookkeeping --
\setcounter{secnumdepth}{3}
\renewcommand{\thesection}{\arabic{section}}
\settocbibname{References}

% -- TOC --
\renewcommand{\cfttoctitlefont}{\Large\bfseries}
\setlength\cftaftertoctitleskip{1em}
\setlength\cftbeforesecskip{0.5em}

\setcounter{page}{1}
\pagenumbering{roman}
\tableofcontents
\clearpage
\setcounter{page}{1}
\pagenumbering{arabic}

% -- Introduction --
\section{Introduction}

Combinatorial optimization is a mathematical field concerned with efficiently
finding certain optimal objects among a large set of objects. A famous example
of a combinatorial optimization problem is the travelling salesman problem, or
TSP\@. Assume a salesman has a list of cities he wants to visit, and knows the
distance between various pairs of cities. The TSP asks: What is the shortest
path which takes the salesman through each city once and back to his starting
point? Here, the optimal object is a shortest path which passes through each
city once and returns the salesman to his point of departure, and the set of
objects is the set of all paths which can be taken.  To see why this problem
could be difficult, assume that every city is connected to every other city and
we want to check all the paths to find the shortest one. If we fix a city of
departure, there are $n-1$ cities left to go to, where $n$ is the total number
of cities.  Each of these choices represents a different path the salesman
could take. In the next city, there are $n-2$ cities left to go to, and so on.
It turns out, given $n$ cities there are $(n-1)!/2$ unique paths which reach
every city and return the salesman to his point of departure. Even for
relatively small values of $n$, the number of possible paths is enormous; far
larger than anything even a computer could hope to check. Consider for example
the choice of $n = 64$. In this case, there are over $10^{86}$ possible paths
to check! More paths than there are atoms in the universe! Using combinatorial
optimization techniques, however, we can significantly reduce the amount of
work needed to find an optimal path.  Indeed, to date, instances of the TSP
have been solved with tens of thousands of cities.

One particular strategy often used in combinatorial optimization is the greedy
optimization strategy, or greedy algorithm. The greedy algorithm attempts to
find an optimal object by making successive locally optimal decisions, in the
hopes that this will lead to a globally optimal solution. For example, a greedy
algorithm for the TSP could work as follows: Fix the city of departure.  To
reach another city, pick the shortest route out of the current city which leads
to an unvisited city. Repeat until you return to the city of departure.
Unfortunately, greedy strategies are often non-optimal. In the case of the TSP,
this particular greedy algorithm will not always yield the optimal path, and in
certain situations, can in fact give the worst possible solution. More
generally, it is known that greedy algorithms are far from optimal for most
instances of the TSP \cite{greedy-tsp}.

%% Need citations here

However, for certain combinatorial optimization problems, greedy algorithms can
be very useful. If they can be proven to be correct, greedy algorithms are
often far simpler to implement than other combinatorial optimization
strategies, such as dynamic programming (see \cite[pp. 378--387]{clrs} for more
information). Furthermore, they often consume less resources when implemented
on a computer and thus run faster and more efficiently. Therefore, knowing when
a greedy algorithm can be applied is very useful.

In this paper, we examine the necessary and sufficient conditions required for
a greedy algorithm strategy to be successful for a given combinatorial
optimization problem. This is done through the study of matroids and their
properties, which as we shall soon see, lead naturally to greedy algorithm
strategies.

The remainder of this paper will proceed as follows. First, we give an example
of a successful greedy algorithm by examining the minimum spanning tree
problem and Kruskal's algorithm. Next we introduce matroids and discuss their
various forms and properties. Then, we link certain properties of matroids to
solutions to a general class of combinatorial optimization problems. Finally,
we show that these solutions can be obtained through greedy strategies.

This paper assumes some familiarity with graph theory and basic linear
algebra. Throughout the paper, we use standard notation from graph theory.
For details, see \cite[pp. 1--6]{matroid-theory}. Note that unless otherwise
indicated, all proofs in this paper are original.

%% Better outline...

\section{The MSTP and Kruskal's Algorithm}

Recall that a \emph{spanning tree} $T$ of a graph $G$ is a connected, acyclic
subgraph of $G$ which contains all the vertices of $G$. Given a weighted graph
$G = (V,E,w)$, the minimum spanning tree problem (MSTP) asks us to find a
spanning tree $T$ of $G$, such that $T$ has minimal weight.

Unlike the TSP, the MSTP can be solved with a simple and elegant algorithm, due
to Kruskal \cite[pp. 631--633]{clrs}.

%Kruskal's algorithm works by always
%adding the lightest unpicked edge to the spanning tree, as long as this edge
%does not create any cycles.

\begin{defn}[Kruskal's Algorithm]
Given a connected graph $G = (V,E,w)$, the minimum spanning tree $T^*$ of
$G$ can be computed as follows:
\begin{enumerate}
    \item Begin with an empty set $T := \emptyset$.
    \item Until $|T| = |V| - 1$, repeat steps 3 and 4 below.
    \item Select a minimum weight edge $e$ in $E - T$, breaking ties
    arbitrarily.
    \item If $T \cup e$ is acyclic, set $T := T \cup e$. Otherwise discard
    the edge $e$ and never consider it again.
    \item Output a minimum spanning tree $T^* = (V,T,w)$ and stop.
\end{enumerate}
\end{defn}

Note that due to our frequent addition and removal of single elements from
sets, we abbreviate $X \cup \{e\}$ to $X \cup e$ and $X - \{e\}$ to $X - e$.

Clearly, Kruskal's algorithm is greedy. At every step, the algorithm
selects a minimum weight edge in $G$, and attempts to add it to the edge set
of the output spanning tree.

\begin{defn}
An algorithm is called a \emph{greedy algorithm} if it makes a locally optimal
decision at each step.
\end{defn}

Kruskal's algorithm shows how efficient greedy algorithms can be, given the
correct problem structure. The algorithm works because the MSTP has two
properties which immediately allow us to identify edges which are part of an
MST of $G$, and edges which are not \cite[pp. 10--11]{eisner}. Let us call a
partition of $G$ into two disjoint sets $A$ and $B$ a cut $(A,B)$ of $G$. The
first property says that an edge $e \in E$ is part of the MST of $G$ if and
only $e$ is the lightest edge across some cut of $G$. This is called the cut
property. The second property says that an edge $e \in E$ is not part of the
MST of $G$ if and only if $e$ is the heaviest edge in some cycle of $G$. This
is called the cycle property.  Because Kruskal's algorithm adds edges from
lightest to heaviest, if an edge $e \in E - T$ is being considered by the
algorithm it must be the lightest edge across some cut, unless it creates a
cycle in $T^* = (V,T,w)$, in which case it is the heaviest edge in some cycle.
Since one of these possibilities always holds, Kruskal's algorithm always
returns a MST of $G$. See \autoref{fig:kruskal-example} for an example of
Kruskal's algorithm in action.

% The point is to try and expose the matroid structure here, where MSTs are
% basis of cycle matroids with lowest weight, which you can find
% by selecting min weight elements and making sure they're independent.

% Old
%Kruskal's algorithm shows how efficient greedy algorithms can be, given
%the correct problem structure. Intuitively, Kruskal's algorithm works because
%given any subset S of the vertex set of a graph G, an MST of the subgraph
%induced by S will always be a part of an MST of G. Furthermore, if an edge
%e is a minimum weight edge across all edges, it must be part of some MST
%of G.
%

\begin{figure}
\centering
\begin{subfigure}[t]{0.4\textwidth}
\centering
\begin{tikzpicture}[scale=0.75,transform shape]
\GraphInit[vstyle=Simple]
\SetGraphUnit{2.5}
\Vertex[a=18,d=2.5]{A}
\Vertex[a=90,d=2.5]{B}
\Vertex[a=162,d=2.5]{C}
\Vertex[a=234,d=2.5]{D}
\Vertex[a=306,d=2.5]{E}
\Edge[label=$4$](A)(B)
\Edge[label=$3$](A)(D)
\Edge[label=$2$](B)(E)
\Edge[label=$5$](C)(D)
\Edge[label=$1$](D)(E)
\Edge[label=$2$,style={pos=0.7}](A)(C)
\Edge[label=$3$,style={pos=0.4}](C)(E)
\Edge[local,label=$1$,color=blue](B)(C)
\end{tikzpicture}
\caption{}
\end{subfigure}
\;
\begin{subfigure}[t]{0.4\textwidth}
\centering
\begin{tikzpicture}[scale=0.75,transform shape]
\GraphInit[vstyle=Simple]
\SetGraphUnit{2.5}
\Vertex[a=18,d=2.5]{A}
\Vertex[a=90,d=2.5]{B}
\Vertex[a=162,d=2.5]{C}
\Vertex[a=234,d=2.5]{D}
\Vertex[a=306,d=2.5]{E}
\Edge[label=$4$](A)(B)
\Edge[label=$3$](A)(D)
\Edge[label=$2$](B)(E)
\Edge[label=$5$](C)(D)
\Edge[label=$2$,style={pos=0.7}](A)(C)
\Edge[label=$3$,style={pos=0.4}](C)(E)
\Edge[local,label=$1$,color=blue](B)(C)
\Edge[local,label=$1$,color=blue](D)(E)
\end{tikzpicture}
\caption{}
\end{subfigure}

\vspace{10pt}
\begin{subfigure}[t]{0.4\textwidth}
\centering
\begin{tikzpicture}[scale=0.75,transform shape]
\GraphInit[vstyle=Simple]
\SetGraphUnit{2.5}
\Vertex[a=18,d=2.5]{A}
\Vertex[a=90,d=2.5]{B}
\Vertex[a=162,d=2.5]{C}
\Vertex[a=234,d=2.5]{D}
\Vertex[a=306,d=2.5]{E}
\Edge[label=$4$](A)(B)
\Edge[label=$3$](A)(D)
\Edge[label=$2$](B)(E)
\Edge[label=$5$](C)(D)
\Edge[label=$3$,style={pos=0.4}](C)(E)
\Edge[local,label=$1$,color=blue](B)(C)
\Edge[local,label=$1$,color=blue](D)(E)
\Edge[local,label=$2$,color=blue,style={pos=0.7}](A)(C)
\end{tikzpicture}
\caption{}
\end{subfigure}
\;
\begin{subfigure}[t]{0.4\textwidth}
\centering
\begin{tikzpicture}[scale=0.75,transform shape]
\GraphInit[vstyle=Simple]
\SetGraphUnit{2.5}
\Vertex[a=18,d=2.5]{A}
\Vertex[a=90,d=2.5]{B}
\Vertex[a=162,d=2.5]{C}
\Vertex[a=234,d=2.5]{D}
\Vertex[a=306,d=2.5]{E}
\Edge[label=$4$](A)(B)
\Edge[label=$3$](A)(D)
\Edge[label=$5$](C)(D)
\Edge[label=$3$,style={pos=0.4}](C)(E)
\Edge[local,label=$1$,color=blue](B)(C)
\Edge[local,label=$1$,color=blue](D)(E)
\Edge[local,label=$2$,color=blue,style={pos=0.7}](A)(C)
\Edge[local,label=$2$,color=blue](B)(E)
\end{tikzpicture}
\caption{}
\end{subfigure}
\caption{%
The execution of Kruskal's algorithm on a weighted graph $G$. Blue edges are
part of the final MST of $G$ generated by the algorithm.
}
\label{fig:kruskal-example}
\end{figure}


%the text below is slightly incorrect, matroids dont generalize properties of
%the MSTP, optimization problems based on matroid theory generalize
%the MSTP

%something along the lines of in next section, we introduce matroids, which will
%allow us reason about a large class of optimization problems with properties
%similar to those of the MSTP, mentioned above. We will then revisit...

In the next section we introduce matroids, which allow us to reason about a
large class of optimization problems with useful properties similar to those of
the MSTP mentioned above. We will revisit Kruskal's algorithm in subsequent
sections and give a more rigorous proof of correctness using matroid theory.

\section{Matroid Theory}

Intuitively, matroids consolidate various ideas of dependence which
arises naturally in certain fields of mathematics, particularly linear
algebra and graph theory. In linear algebra, a set of vectors $U$ of a
vector space $V$ is called dependent if there is a linear of combination of
these vectors which sum to $0$. In graph theory, a set of edges $S$ of a
graph $G$ is called dependent if the edges of $S$ contain a cycle.

An interesting property of matroids is that there are several equivalent
axiom systems which can be used to reason about them. In this essay, we
will only explore two such axiom systems: the independent set system, and
the bases system. For a more thorough overview of matroid theory, see
\cite{matroid-theory}.

\begin{defn}
A \emph{matroid} $M$ is an ordered pair $(E, \I)$, where $E$ is a finite
set and $\I$ is a collection of subsets of $E$ satisfying the
following three conditions:
\begin{description}
    \item[(I1)] $\emptyset$ is in $\I$.
    \item[(I2)] If $I \in \I$ and $I' \subseteq I$, then $I' \in
                \I$.
    \item[(I3)] If $I_1$ and $I_2$ are in $\I$ and $|I_1| < |I_2|$,
                then there exists an element $e \in I_2 - I_1$ such that
                $I_1 \cup e \in \I$.
\end{description}
\end{defn}

We call $E$ the \emph{ground set} of a matroid $M = (E, \I)$ and we call
the members of $\I$ the \emph{independent sets} of $M$. A subset $S$ of $E$
which is not in $\I$ is called a \emph{dependent set}. We also call (I2)
the \emph{hereditary property} and (I3) the \emph{exchange property}.

From this definition, we can already intuitively see that matroids encapsulate
lots of structure which is important when solving combinatorial optimization
problems. Specifically, (I2) and (I3) are reminiscent of the overlapping
subproblem structure and optimal substructure exploited by dynamic programming
algorithms \cite{clrs}. These conditions imply that we are guaranteed that all
subsets of an independent set are themselves independent, and that we can
always build larger independent sets from smaller ones\footnotemark.

\footnotetext{Until we build a maximal independent set, at which point no more
elements can be exchanged via (I3).}

\begin{prp}\label{graphs-are-matroids}
Let $G = (V,E)$ be a graph and let $\I$ be the set of subsets of $E$ which
do not contain a cycle. Then $M = (E,\I)$ is a matroid.

\begin{proof}
By definition, $\emptyset \in \I$ and thus $\I$ satisfies (I1). It is evident
that if $H \in \I$ and $K \subseteq H$, then $K \in \I$. Thus $\I$ satisfies
(I2). We show that $\I$ satisfies (I3) by contradiction. Suppose $I_1$ and
$I_2$ are elements of $\I$, and $|I_1| < |I_2|$. Assume that for all $e \in
I_2 - I_1$, the set $I_1 \cup e \not\in \I$. This means that the set $I_1
\cup e$ contains a cycle for all $e$. Since $I_1$ is itself acyclic, there
is only one cycle in $I_1 \cup e$, and this cycle must contain $e$. Now,
consider $I_1 \cup I_2$. This set has exactly $|I_2 - I_1|$ cycles, one for
each $e \in I_2 - I_1$. Let $I_3$ be a maximal independent set in
$I_1 \cup I_2$, obtained by removing an edge from each of the $|I_2 - I_1|$
cycles in $I_1 \cup I_2$. Now,
\begin{align*}
    |I_3| &= |I_1 \cup I_2| - |I_2 - I_1|\\
    &= |I_1| + |I_2| - |I_1 \cap I_2| - |I_2 - I_1|\\
    &= |I_1| + |I_2| - |I_1 \cap I_2| - (|I_2| - |I_1 \cap I_2|)\\
    &= |I_1|.
\end{align*}
It follows that $|I_3| < |I_2|$, contradicting the maximality of $I_3$.
Therefore, (I3) holds, and $M$ is indeed a matroid.
\end{proof}
\end{prp}

A matroid derived from a graph $G$ is called a \emph{cycle matroid} and is
denoted $M(G)$. See \autoref{fig:cycle-matroid-example} for an example of a
cycle matroid.

\begin{figure}
\centering
\begin{tikzpicture}[scale=0.75,transform shape]
\GraphInit[vstyle=Simple]
\SetGraphUnit{2.5}
\Vertex[a=45,d=2.5]{A}
\Vertex[a=135,d=2.5]{B}
\Vertex[a=225,d=2.5]{C}
\Vertex[a=315,d=2.5]{D}
\Edge[label=$e_1$](A)(B)
\Edge[label=$e_2$](B)(C)
\Edge[label=$e_4$](B)(D)
\Edge[label=$e_3$](C)(D)
\end{tikzpicture}

\caption{Consider the cycle matroid $M(G)$ of the graph above. The collection
of independent sets of $M(G)$ are exactly the subsets of the edge set of $G$
which do not contain any cycles. Thus $\I = \{$$e_1,e_2,e_3,e_4,e_5,$$\{e_1,e_2\},$
$\{e_1,e_3\},$$\{e_1,e_4\},$$\{e_2,e_3\},$$\{e_2,e_4\},$$\{e_3,e_4\},$$\{e_1,e_2,e_3\},$$
\{e_1,e_2,e_4\},$$\{e_1,e_3,e_4\}$$\}$, where $\I$ is the independent set of
$M(G)$. Note that $\{e_2,e_3,e_4\} \not\in \I$ as the edges $e_2$, $e_3$ and
$e_4$ form a cycle in $G$.}
\label{fig:cycle-matroid-example}
\end{figure}


Consider a collection $\B$ of maximal independent sets of a matroid $M$.
Given the ground set $E$ of $M$ we can easily reconstruct the independent
sets of $M$ as follows: The set $X$ is in $\I$ if and only if $X \subseteq
B$ for some $B \in \B$. We call a maximal independent set of a matroid a
\emph{base} or \emph{basis}.

From our discussion of MSTs and \autoref{graphs-are-matroids} we can deduce
that any spanning tree of a graph $G$ is a basis of the cycle matroid $M(G)$.
Notice that this idea of a basis coincides with our idea of a basis from linear
algebra. Like bases of a vector space, all spanning trees are equicardinal.
Furthermore, if one ``fills in the cycles'' we can imagine that a spanning tree
``spans'' the graph, just as the span of a basis spans a vector space.

\begin{prp}\label{bases-are-equicardinal}
Let $\B$ be a collection of bases of a matroid $M = (E,\I)$. Then the elements
of $\B$ are equicardinal.

\begin{proof}
Let $B_1$ and $B_2$ be elements of $\B$. Suppose $|B_1| < |B_2|$. Then by
(I3), there exists a element $e \in B_2 - B_1$ such that $B_1 \cup e \in \I$.
However, this contradicts the maximality of $B_1$. Therefore,
$|B_1| \geq |B_2|$. Similarly, $|B_2| \geq |B_1|$. Thus $|B_1| = |B_2|$. Since
$B_1$ and $B_2$ were chosen arbitrarily, all elements of $\B$ are equicardinal.
\end{proof}
\end{prp}

The bases of a matroid $M$ also have a very attractive property, similar to the
exchange property for independent sets, called the
\emph{basis exchange property}.

\begin{prp}\label{basis-exchange}
Let $\B$ be a collection of bases of a matroid $M = (E,\I)$. If $B_1$ and
$B_2$ are elements of $\B$ and $x \in B_1 - B_2$, then there exists an element
$y \in B_2 - B_1$ such that $(B_1 - x) \cup y \in \B$.

\begin{proof}
Consider $B_1 - x \subset B_1$. Then $B_1 - x$ is independent by (I2). Now,
$|B_1 - x| < |B_2|$ by \autoref{bases-are-equicardinal}. Thus, there exists an
element $y \in B_2 - B_1$ such that $(B_1 - x) \cup y \in \I$ by (I3). Since
$|(B_1 - x) \cup y| = |B_1|$ and all bases are equicardinal, there cannot
exist an independent set containing $(B_1 - x) \cup y$. Thus $(B_1 - x) \cup
y \in \B$.
\end{proof}
\end{prp}
%% Lemmas instead of propositions?

%% For completness, should be noted that bases characterize matroids if
%% B satisfies (B1) and (B2).

In the next section, we link the bases of matroids to solutions to a general
class of combinatorial optimization problems. We then explore solutions to to
these problems using greedy strategies.

\section{Greedy Algorithms and Matroids}

Let us now define more precisely what it means for a problem to be a
combinatorial optimization problem. As we mentioned earlier, combinatorial
optimization problems deal with finding an optimal object among a large set of
objects.

\begin{defn}
Let $\I$ be a collection of subsets of a set $E$. Define $w$ as a function from
$E$ to $\mathbb{R}^{\geq 0}$, where
\begin{align*}
w(X) &= \sum_{x \in X}{w(x)}
\end{align*}
for all $X \subseteq E$. We call $w(X)$ the \emph{weight} of $X$.

An \emph{optimization problem} for a triple $(E,\I,w)$ is the problem of finding a
maximal element $S$ of $\I$ such that $S$ has minimal weight.
\end{defn}

We call a maximal element with minimum weight a \emph{solution} to an
optimization problem. If an algorithm always finds a solution to a
combinatorial optimization problem $P$, we say the algorithm solves $P$.

It is easy to see how this definition encapsulates our intuition about
combinatorial optimization problems. Because $\I$ is a collection of subsets of
a set, it is plausible that $\I$ could become exponential in $E$, and thus very
large. Specifically, in the worst case if $\I = \mathcal{P}(E)$, the power set
of $E$, then $|\I| = 2^{|E|}$, which grows exponentially in the size of $E$.
This means that finding a maximal element $S$ of $I$ with minimal weight could
become exponentially difficult.

For example, the TSP corresponds to the optimization problem where $\I$ is a
collection of subsets of the edge set of a graph $G = (V,E,w)$ which do not
form cycles or are Hamiltonian cycles\footnotemark, and $w$ is simply the
weight function of $G$. In this optimization problem, a maximal element of $\I$
would be a Hamiltonian cycle, and a solution to $(E,\I,w)$ would be Hamiltonian
cycle of minimal weight.

\footnote{A Hamiltonian cycle of a graph $G$ is a cycle of $G$ which touches
every node in the graph exactly once.}

Given this definition of a combinatorial optimization problem, we can now
define a general greedy algorithm which attempts to find a solution to
arbitrary combinatorial optimization problems.

\begin{defn}[Generalized Greedy Algorithm]
Given a combinatorial optimization problem $P = (E,\I,w)$, a potential solution
to $P$ can be computed as follows:

\begin{enumerate}
    \item Begin with an empty set $X := \emptyset$.
    \item Select a minimum weight element $e \in E - X$, such that $X \cup e
          \in \I$. If such an element exists, set $X := X \cup e$. Otherwise,
          output $X$ and stop.
    \item Repeat the previous step until the algorithm stops.
\end{enumerate}
\end{defn}

We abbreviate the generalized greedy algorithm to GGA.

Clearly, the GGA is greedy. At every step, the algorithm makes a locally
optimal decision by selecting an element of minimum weight. Since $E$ is
finite and the algorithm adds one element in $E - X$ to $X$ at every step,
the algorithm must stop in a finite number of steps. Furthermore, because
$X$ is only augmented when $X \cup e$ is an element of $\I$, the output $X$
is also an element of $\I$. Thus, it is plausible that the GGA solves some
combinatorial optimization problems.

Incredibly, it can be shown that if $(E,\I)$ is a matroid, then the solution to
the combinatorial problem $(E,\I,w)$ is given by the output of the generalized
greedy algorithm! This follows from the fact that $X$ can always be augmented
using the exchange property until it becomes a basis in $\I$, and that the
algorithm only selects minimum weight elements to add to $X$.

\begin{thm}\label{GGA-solves-matroids}
Let $M = (E,\I)$ be a matroid, and let $P = (E,\I,w)$ be a combinatorial
optimization problem. Then the generalized greedy algorithm solves $P$.

\begin{proof}

Let $X$ be the output of the GGA given $P$. We show that $X$ is a maximal
element of $\I$. At every step of the algorithm, $|X| < |E|$. If $|X| = |E|$,
then the algorithm must terminate, as $E - X$ would be empty. Furthermore, at
every step of the algorithm, $X$ is independent. This is because $X$ is
initialized to the empty set, and at every subsequent step $X$ is augmented
such that $X$ remains in $\I$. Therefore, by (I3), we can always increase the
cardinality of $X$ by one in each step of the algorithm. Suppose the bases of
$M$ have cardinality $r$.  Since the cardinality of $X$ is initially $0$, and
at every step we increase the cardinality of $X$ by one, $X$ will be a basis of
$M$ after $r$ steps. Thus, the output of the GGA is a maximal element of $\I$.

Now, let $Y$ be a maximal element of $\I$ with minimal weight, i.e.\ a solution
to $P$. We show that $X$ has minimal weight. Since both $X$ and $Y$ are bases
of $\I$, we can exchange an element $x \in X - Y$ with an element $y \in Y - X$
to form a new basis $(X - x) \cup y$ by \autoref{basis-exchange}. This new
basis has weight lower than the weight of $X$ if and only if $w(y) < w(x)$. We
show this to be false. Let $X' \subset X$ be the elements of $X$ chosen before
$x$ was chosen. Because the GGA chose $x$, this implies $x$ had lower weight
than any other element in $E - X'$. However, $Y - X \subset Y - X' \subset E -
X'$. Thus, as $y \in Y - X$ it follows that $w(x) \leq w(y)$. Therefore, $w(X)
\leq w((X - x) \cup y)$. Let $X_1 = (X - x) \cup y$. We can form a chain of
alternate bases $X_{i+1}$ of $\I$ by exchanging elements between $X_i$ and $Y$,
where $i \in \{1,2,\ldots,k\}$. After every exchange, we can use a similar
argument to show that $w(X_i) \leq w(X_{i+1})$. This process will stop when we
have exchanged all the elements of $X - Y$ and $Y - X$, at which point $X_k =
Y$. Now, since $w(X_i) \leq w(X_{i+1})$ at every link in the chain, this
implies
\begin{align*}
    w(X) \leq w(X_1) \leq w(X_2) \leq \cdots \leq w(X_{k - 1}) \leq w(Y).
\end{align*}
However, this contradicts the minimality of $Y$, and thus an equality must hold
throughout. Therefore $w(X) = w(Y)$, and $X$ is indeed of minimal weight. Since
$X$ is a maximal element of $\I$ with minimal weight it is a solution to $P$,
and the GGA solves $P$.
\end{proof}
\end{thm}

% fix the wording of the last line above

Let us now reconsider Kruskal's algorithm and the MSTP in the context of
combinatorial optimization problems. Fix a connected graph $G = (V,E,w)$. We
can think of the MSTP as an optimization problem $(E,\I,w)$, where $\I$ is a
collection of acyclic subsets of the edge set $E$ of $G$, and $w$ is simply the
weight function of $G$. This problem also has a natural matroid structure,
given by the cycle matroid $M(G)$ of $G$. Let us now compare Kruskal's
algorithm to the GGA. If we can show that Kruskal's algorithm is somehow a
special case of the GGA, its correctness comes for free by the previous
theorem. For every edge $e$ added to the edge set $T$, Kruskal's algorithm
checks to make sure $T \cup e$ is acyclic. This corresponds exactly to the
independence check, $X \cup e \in \I$, executed by the GGA, where $\I$ is the
independent set of $M(G)$. It is simple to see that the other aspects of the
algorithm are also essentially equivalent. Thus, we can view Kruskal's
algorithm as a special case of the GGA for cycle matroids. It follows from
\autoref{GGA-solves-matroids} that Kruskal's algorithm solves the MSTP for $G$.

\begin{cor}\label{kruskal-solves-MSTP}
Let $G = (V,E,w)$ be a connected graph. Then Kruskal's algorithm solves the
MSTP for $G$.
\end{cor}

\begin{defn}
We say a combinatorial optimization problem $P = (E,\I,w)$ has \emph{matroid structure}
if $M = (E, \I)$ is a matroid.
\end{defn}

So far, we have shown that the GGA solves all combinatorial optimization
problems with an underlying matroid structure. It is natural to now ask what
other combinatorial optimization problems, if any, the GGA is capable of
solving. Perhaps counterintuitively, it can be shown that the GGA solves a
combinatorial optimization problem if and only if the problem has a matroid
structure.

\begin{thm}\label{GGA-solves-only-matroids}
Let $P = (E,\I,w)$ be a combinatorial optimization problem. Then $P$ has matroid
structure if and only if $P$ satisfies the following conditions:
\begin{description}
    \item[(I1)] $\emptyset$ is in $\I$.
    \item[(I2)] If $I \in \I$ and $I' \subseteq I$, then $I' \in
                \I$.
    \item[(G)]  The generalized greedy algorithm solves $P$ for all weight
                functions $w:E \to \mathbb{R}^{\geq 0}$.
\end{description}
\begin{proof}

Assume $P$ has matroid structure. Then $P$ satisfies (I1) and (I2) by
definition. By \autoref{GGA-solves-matroids}, $P$ also satisfies (G).

Conversely, assume $P$ satisfies (I1), (I2) and (G). We show by contradiction
that $P$ satisfies (I3) also, and thus has matroid structure. Let $I_1$ and
$I_2$ be elements of $\I$ such that $|I_1| < |I_2|$ and for all $x$ in $I_2 -
I_1$, $I_1 \cup x \not\in \I$. We proceed by constructing a weight function $w:
E \to \mathbb{R}^{\geq 0}$ under which the GGA fails to solve $P$.

Let $r$ be the cardinality of a maximal element in $\I$ and let
$w: E \to \mathbb{R}^{\geq 0}$ be a weight function defined by
\begin{equation*}
    w(x) = \left\{
            \begin{array}{ll}
                0 & \quad \text{if } x \in I_1\\
                \dfrac{1}{2|I_2 - I_1|} & \quad \text{if } x \in I_2 - I_1\\
                1 & \quad \text{if } x \in E - (I_1 \cup I_2)
            \end{array}
            \right.
\end{equation*}

Now, let $S$ be the solution to $P$ returned by the GGA. Given the weight
function $w$ above, the GGA would first add all of $I_1$ to $S$. Next,
the GAA would attempt to add the elements of $I_2$ to $S$, however, because
$I_1 \cup x$ is dependent for all $x$ in $I_2$, the GGA would be forced to
add the elements of $E - (I_1 \cup I_2)$ to $S$ instead. The solution
returned by the GGA would therefore have weight
\begin{equation*}
    w(S) = 0(|I_1|) + 1(r - |I_1|) = r - |I_1|.
\end{equation*}

We now show that there exists a maximal element $S^*$ of $\I$ with weight lower
than that of $S$. Let $S^* = (I_1 \cap I_2) \cup (I_2 - I_1) \cup X$, where
$X$ is an arbitrary subset of $E - (I_1 \cup I_2)$ and $|X| = r - |I_2|$. The
potential solution $S^*$ has weight
\begin{align*}
    w(S^*) &= 0(|I_1 \cap I_2|) + \left(\frac{1}{2|I_2 - I_1|}\right)|I_2 - I_1| + 1(r - |I_2|)\\
    &= 1/2 + r - |I_2|
\end{align*}

Observe that because $|I_1|$ and $|I_2|$ can only take discrete values,
$|I_2| - |I_1| \geq 1$. Now, we find that the difference $w(S) - w(S^*)$,
\begin{align*}
    w(S) - w(S^*) &= (r - |I_1|) - (1/2 + r - |I_2|)\\
    &= |I_2| - |I_1| - 1/2\\
    &\geq 1/2
\end{align*}
is bound from below by $1/2$. Thus $w(S^*) < w(S)$.

However, this contradicts the minimality of $S$ as per condition (G).
Therefore, (I3) holds and $P$ has matroid structure.
\end{proof}
\end{thm}

This result is interesting for several reasons.

First, it tells us something important about how much structure a black-box
combinatorial optimization problems must have to be soluble. If we think of the
generalized greedy algorithm as the simplest general method of solving
combinatorial optimization problems, we see informally that it is in general
very difficult to solve combinatorial optimization problems without knowing
more about their internal structure. Indeed, unless we require that the
problems have matroid structure, even the simplest general method will not be
able to solve them.

Second, this result allows us to quickly identify combinatorial optimization
problems which can be solved using greedy strategies. When faced with a novel
combinatorial optimization problem, this allows us to quickly determine
whether it is worth pursuing more complex computational solutions or not.

From an applied perspective, this result is important because it establishes
that the generalized greedy algorithm solves every combinatorial optimization
problem with matroid structure, and thus, one does not need to optimize several
algorithms to solve different kinds of combinatorial optimization problems
quickly.

Finally, we see that \autoref{GGA-solves-only-matroids} provides a complete
characterization of all combinatorial optimization problems soluble using
greedy strategies. Namely, a combinatorial optimization problem is soluble
if and only if it has a matroid structure.

% note that we can actually prove a slighly stronger result: there exist an
% infinite family of weight functions for which the GGA fails given (I3)
% fails

%% Intuitively, you want to say that it is natural that the GGA only solves
%% problems with matroid structure because:
%%      1. The alogorithm is weight function invariant
%%      2. The algorithm is greedy, myopic, in a sense, it can't make
%%          bad choices and change them retroactively.

%% To add to below, potentially talk about how the GGA and the matroid fix
%% the combinatorial optimization problem, and then how the GGA and the matroid
%% fix the combinatorial optimization problem.
%% In a way, we can think that the GGA and the comb. opt. prblm. fix the matroid,

\section{Conclusions and Further Discussion}

In conclusion, we have completely characterized the combinatorial optimization
problems which are soluble using greedy algorithm strategies. We have proven
that greedy algorithm strategies solve combinatorial optimization problems if
and only if these problems possess a matroid structure.

Of course, there are many more features of greedy strategies that we could explore
in combinatorial optimization which are interesting from a combinatorial
perspective as well as an applied perspective.

One important question we can ask, for example, is how well the generalized
greedy algorithm approximates the true solution to combinatorial optimization
problems without matroid structure. For example, it is known that the greedy
algorithm provides a fairly good solution to the knapsack problem in the general
case.

The generalized greedy algorithm is also often used to establish good initial
conditions before a more specific algorithm is used to solve a combinatorial
optimization problem. For example in graph colouring, a greedy strategy is often
used to establish initial conditions before local search techniques are used.

For a more in depth overview of greedy algorithm theory and matroid theory,
we refer the reader to \cite{clrs} and \cite{matroid-theory}.

\clearpage
\bibliographystyle{plain}
\bibliography{doc.bib}

\end{document}
